\documentclass[a4paper, fleqn]{article}

\usepackage{amsmath}
\usepackage{enumitem}
\usepackage{graphicx}

\begin{document}

\title{Capstone I - Journal 01 \\ Project 61 $\cdot$  Steelcase \\ Reduction of Logistics and Packaging Costs}
\author{Basil R. Yap}
\date{2018 February 02}
\maketitle

\section{Summary}

Considering the project is at its infancy, the goal I set out to accomplish within the first two weeks are as follows:
\begin{enumerate}[label=(\alph{*})]
\item Set up the necessary tools/documentation needed for the efficient execution of tasks performed by members of the team or by the team as whole.
\item Conduct a precedent analysis into the background of the company and their current solutions for the proposed problem.
\item Conduct initial investigations into possible tools and methods needed to solve supply chain and logistic management problems. 
\end{enumerate} 
The manifestation of these goals would serve as the foundation of future tasks, streamlining documentation/work assignment/communication within the team while solidifying my personal knowledge of the project and the tools available to tackle the task.


\section{Problems Encountered}

The goals I chose were a product of my prior experience working on Industry Projects. At the start of such a project, I have Identified the following problems that might arise immediately or in the future:
\begin{itemize}
\item \textbf{Inaccessibility and Disarray of documents shared among the team and stakeholders}\\
Documents created and shared are not stored/filed in a systematic manner, causing members to not being able to locate essential information or possessing incomplete information.
\item \textbf{Inconsistency of structure and styling of documents}\\
Documents created by various members of the team vary in format and content structure, causing members to be more likely to miss out on essential information and reflects badly on the coordination/professionalism of the team.
\item \textbf{Overlapping task assignment among team}\\
Coordination between team members on the assignment of tasks are not clearly documented. Team members may accidentally work on the same task simultaneously, wasting effort that might be better applied elsewhere.
\item \textbf{Lack of Information regarding the current solutions and measures adopted by Steelcase with regards to packaging and shipping of goods}\\
The materials and method of shipment employed by Steelcase is not known yet. Prior knowledge pertaining to such areas would be needed for the first meeting with Steelcase.
\item \textbf{Ambiguity on the nature of the data supplied by Steelcase}\\
The structure and content of the data provided by Steelcase is not known yet. It is important for us to research on the possible tools in order to adapt to the problem given.
\end{itemize}

\section{Actions Taken}

Based on the problems faced, the following actions were taken by me:
\begin{itemize}
\item Create file directory system to organize files shared on Google Drive
\item Create templates for common use documents such as Minutes and Emails
\item Introduce task assignment application, Trello, to implement SCRUM system
\item Research on the environmental impact of Steelcase via various self-published reports
\item Research and document various tools that I have encountered that might be useful in the processing, modeling and visualization of the data provided
\end{itemize}

In addition to that, I have taken up the role as the Point of Contact between Steelcase and my team.

\section{Insights Obtained}

\subsection{Management Systems}

One of the major lessons I learned from ESD is the importance of designing management systems. Projects, especially large-scale ones like Capstone, will fail at its seams if not enough preparation is done on the part of Systems Engineer.\\
\\
For a project that spans multiple departments, i.e. ESD and EPD, it is easy for us to work on our respective halves of our project and come together at the end to complete the product, this method is very prone to miscommunication and error.

\pagebreak

I hope that by implementing a system like SCRUM, this problem would be minimized. SCRUM is a task management tool usually used by fast-pace software development teams. It divides tasks into four distinct groups, To-do, In Progress, Done and Postponed. The table is open to the entire team, usually allowing team members to pick up vacant tasks to complete. This affords a clear assignment of tasks while encouraging competitiveness among team members.

\subsection{Data Tools}

From prior experience, the tools we could use are as follows:
\begin{enumerate}
\item \textbf{Python} \begin{itemize}
\item General programming language
\item Can do most tasks given the correct library
\item Can handle large amounts of data
\item Usable for implementation but not encouraged
\item Notable packages: Tensorflow, scrapy, urllib
\end{itemize}
\item \textbf{R Language}  \begin{itemize}
\item General programming language
\item Can handle most tasks given the correct library
\item Adept at statistical computation
\item Has good visualization capabilities
\item Good for Proof of Concept tests
\item Constrained by local memory size
\item Notable packages: ggplot2, forecast
\end{itemize}
\item \textbf{AMPL}  \begin{itemize}
\item Symbolic programming language
\item Adept at optimization computation
\item Requires constant model structure
\item Has API for Python and R
\end{itemize}
\item \textbf{Tableau} \begin{itemize}
\item Data visualization platform
\item Adept at data visualization and exploratory analysis
\item Has Limited functionality
\end{itemize}
\end{enumerate}

\pagebreak

\section{Timeline}

The my personal timeline for next three weeks are as follows:
\begin{itemize}
\item \textbf{Week 3:}    \begin{enumerate}
\item Acquire and Clean data (priority: Sales data)
\item Conduct exploratory analysis of data
\item Conduct precedent analysis of materials and form for packaging boxes
\end{enumerate}
\item \textbf{Week 4:}     \begin{enumerate}
\item Rank importance of product line depending on metrics such as profitability
\item Continue Cleaning data (If applicable)
\item Conduct trend analysis for Shipping data (First hypothesis/model)
\item Begin simple data visualization
\end{enumerate}
\item \textbf{Week 5:}    \begin{enumerate}
\item Create metrics for Validation
\item Validate models
\item Create more models
\item Prepare data visualizations for Review I
\end{enumerate}
\end{itemize}

\end{document}