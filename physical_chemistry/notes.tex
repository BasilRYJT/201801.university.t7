\documentclass[a4paper, fleqn]{article}

\usepackage{amsmath}
\usepackage{enumitem}
\usepackage{graphicx}

\begin{document}

\title{Notes \\ Introduction to Physical Chemistry}
\author{Basil R. Yap}
\date{2018 January}
\maketitle

\section{Week 1}

Dalton's Atomic Theory: \begin{itemize}
\item atoms are indestructible particles \begin{itemize}
\item disproved by nuclear fission/fusion
\item true in chemical reactions
\end{itemize}
\item all atoms of a given element are identical and only distinguishable via atomic weight
\item compounds are whole number ratios of atoms
\end{itemize}

From Dalton's Postulates: \begin{itemize}
\item Law of Conservation of Mass: mass is not created or destroyed in a chemical reaction
\item Law of Definite Proportion: compounds always contain the same proportion of elements
\item Law of Multiple Proportions: the ratio of masses in a compound are whole numbers
\end{itemize}

Classical Atomic Models:\begin{itemize}
\item Rutherford and Bohr proposed electrons move in an orbit
\item Bohr had explanation for color emissions: \begin{itemize}
\item orbits have energy levels based on distance from nucleus
\item moving from one orbit to another releases energy corresponding to light
\item impossible for electron to exist between orbits
\end{itemize}
\end{itemize}

Failure of Classical model: \begin{itemize}
\item based on classical mechanics
\item electrons in motion similar to classical physics \begin{itemize}
\item electron should lose energy over time
\end{itemize}
\item predicted emission patterns of hydrogen, but not other elements
\end{itemize}
Another atomic model needed: basis with wave-particle duality of matter

\end{document}