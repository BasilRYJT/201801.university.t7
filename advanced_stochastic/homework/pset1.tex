\documentclass[a4paper, fleqn]{article}

\usepackage{amsmath}
\usepackage{enumitem}
\usepackage{graphicx}

\begin{document}

\title{Homework I \\ Advanced Topics in Stochastic Modelling}
\author{Basil R. Yap}
\date{2018 March 25}
\maketitle

\section{Question 1}

Let $X_T$ be the fraction of time that it rains given $T$ days have elapsed.\\
Where,
$$\begin{aligned}
X_T&=\frac{r_1+r_2+\cdots+r_n}{T}\\
\sum_i^n r_i&+\sum_i^m d_i=T\\
r_i&\text{ is the length of }i\text{th raining spell}\\
d_i&\text{ is the length of }i\text{th dry spell}
\end{aligned}$$\\
The long-run fraction of time that it rain can be expressed by:
$$
\begin{aligned}
\lim_{T\rightarrow\infty}X_T&=\lim_{T\rightarrow\infty}\frac{\sum_i^n r_i}{T}\\
&=\frac{E[r_i]}{E[T_i]}\\
&=\frac{E[r_i]}{E[r_i]+E[d_i]}\ \ \ \ \#\ r_i\perp d_i\\
&=\frac{2}{2+7}\\
&=\frac{2}{9}\\
\text{where }T_i=r_i+d_i\text{ , the period}&\text{ of the }i\text{th renewal cycle of the system.}
\end{aligned}
$$
\pagebreak
\section{Question 2}

\subsection{Part a}

Let $X_T$ be the fraction of people who go to Hilton Hotel given $T$ hours have elapsed.\\
Where,
$$\begin{aligned}
X_T&=\frac{y_1+y_2+\cdots+y_n}{P_T}\\
\sum_{i=1}^n y_i&+\sum_{i=1}^m n_i=P_T\ \ \ \ \#\ y_i=7\ \forall i\\
y_i\text{ is the number of people who }&\text{do go to Hilton Hotel during the }i\text{th renewal cycle}\\
d_i\text{ is the number of people who }&\text{go elsewhere during the }i\text{th renewal cycle}\\
\end{aligned}$$\\
The long-run fraction of people who go to Hilton Hotel can be expressed by:
$$
\begin{aligned}
\lim_{T\rightarrow\infty}X_T&=\lim_{T\rightarrow\infty}\frac{\sum_i^n y_i}{P_T}\\
\text{as }T\rightarrow\infty\ &,\ P_T\rightarrow\infty\\
&=\lim_{P_T\rightarrow\infty}\frac{\sum_i^n y_i}{P_T}\\
&=\frac{E[y_i]}{E[T_i]}\\
&=\frac{7}{7+\frac{36}{60}\cdot10}\ \ \ \ \#\ \text{proportionality property}\\
&=\frac{7}{13}\\
\text{where }T_i=y_i+n_i\text{ , the number of}&\text{ people in the }i\text{th renewal cycle.}
\end{aligned}
$$
\pagebreak
\subsection{Part b}

Let $X_i$ be amount of time spent by the $i$th person to reach Hilton Hotel at each renewal cycle.\\
Let $Y_i$ be the inter-arrival time of the $i$th person and $i-1$th person  \\
Where,
$$
\begin{aligned}
Y_i&\sim\text{Exp}(10)\ \ \ \ \forall i\ \ \ \ \#\ \text{independent increment property}\\\\
X_i&=\frac{36}{60}+\sum_{j=i+1}^7Y_j\ \ \ \ \forall i \setminus\{7\}\\
X_7&=\frac{36}{60}
\end{aligned}
$$
The expected time spent by people going to Hilton Hotel can be expressed by:
$$
\begin{aligned}
E[X]&= \sum_{i=1}^7\left(\frac{1}{7}\cdot\sum_{j=i+1}^7Y_j\right)\ \ \ \ \#\ \text{proportionality property}\\
&=\frac{1}{7}E\left[\sum_{i=2}^7Y_i\right]+\frac{1}{7}E\left[\sum_{i=3}^7Y_i\right]+\frac{1}{7}E\left[\sum_{i=4}^7Y_i\right]\\&+\frac{1}{7}E\left[\sum_{i=5}^7Y_i\right]+\frac{1}{7}E\left[\sum_{i=6}^7Y_i\right]+\frac{1}{7}E\left[\sum_{i=7}^7Y_i\right]+\frac{1}{7}E[36]\\
&=\frac{6}{7}E[Y_7]+\frac{5}{7}E[Y_6]+\frac{4}{7}E[Y_5]+\frac{3}{7}E[Y_4]+\frac{2}{7}E[Y_3]+\frac{1}{7}E[Y_2]\\
&= \frac{36}{60}+\frac{6}{7}\cdot\frac{1}{10}+\frac{5}{7}\cdot\frac{1}{10}+\frac{4}{7}\cdot\frac{1}{10}+\frac{3}{7}\cdot\frac{1}{10}+\frac{2}{7}\cdot\frac{1}{10}+\frac{1}{7}\cdot\frac{1}{10}\\
&=\frac{9}{10}\ \ (\textit{hours})
\end{aligned}
$$
\pagebreak
\section{Question 3}

\subsection{Part a}

Let $X_T$ be the fraction of shots thrown by any child given $T$ is the total number of shots thrown.\\
Where,
$$\begin{aligned}
X_T&=\frac{a_1+a_2+\cdots+a_n}{T}\\
\sum_{i=1}^n a_i&+\sum_{i=1}^m b_i+\sum_{i=1}^kc_i=T\\
a_i&\sim Geo(p_1)\\
b_i&\sim Geo(p_2)\\
c_i&\sim Geo(p_3)\\
a_i\text{ is the number of shots}&\text{ thrown by child a at the }i\text{th renewal cycle}\\
b_i\ \&\ c_i\text{ are the number of shots}&\text{ not thrown by child a at the }i\text{th renewal cycle}
\end{aligned}$$\\
The long-run fraction of shots thrown by any child can be expressed by:
$$
\begin{aligned}
\lim_{T\rightarrow\infty}X_T&=\lim_{T\rightarrow\infty}\frac{\sum_i^n a_i}{T}\\
&=\frac{E[a_i]}{E[T_i]}\\
&=\frac{E[a_i]}{E[a_i]+E[b_i]+E[c_i]}\ \ \ \ \#\ a_i\perp b_i\perp c_i\\
&=\frac{\frac{1}{p_1}}{\frac{1}{p_1}+\frac{1}{p_2}+\frac{1}{p_3}}\\
&=\frac{\frac{1}{p_1}}{\frac{p_2p_3+p_1p_3+p_1p_2}{p_1p_2p_3}}\\
&=\frac{p_2p_3}{p_2p_3+p_1p_3+p_1p_2}\\
\text{where }T_i=a_i+b_i+c_i\text{ , the period}&\text{ of the }i\text{th renewal cycle of the system.}
\end{aligned}
$$
This solution can be applied to any of the three children by reassigning the values of $p_1,\ p_2\ \&\ p_3$
\pagebreak
\subsection{Part b}
$p_1=\frac{2}{3}\ ,\ p_2=\frac{3}{4}\ ,\ p_3=\frac{4}{5}$
\begin{enumerate}[label=\textbf{For Child \Alph*}]
\item $$
\begin{aligned}
\lim_{T\rightarrow\infty}X_T&=\lim_{T\rightarrow\infty}\frac{\sum_i^n a_i}{T}\\
&=\frac{p_2p_3}{p_2p_3+p_1p_3+p_2p_3}\\
&=\frac{\frac{12}{20}}{\frac{12}{20}+\frac{8}{15}+\frac{6}{12}}\\
&=\frac{18}{49}
\end{aligned}
$$
\item $$
\begin{aligned}
\lim_{T\rightarrow\infty}X_T&=\lim_{T\rightarrow\infty}\frac{\sum_i^n b_i}{T}\\
&=\frac{p_1p_3}{p_2p_3+p_1p_3+p_2p_3}\\
&=\frac{\frac{8}{15}}{\frac{12}{20}+\frac{8}{15}+\frac{6}{12}}\\
&=\frac{16}{49}
\end{aligned}
$$
\item $$
\begin{aligned}
\lim_{T\rightarrow\infty}X_T&=\lim_{T\rightarrow\infty}\frac{\sum_i^n c_i}{T}\\
&=\frac{p_1p_2}{p_2p_3+p_1p_3+p_2p_3}\\
&=\frac{\frac{6}{12}}{\frac{12}{20}+\frac{8}{15}+\frac{6}{12}}\\
&=\frac{15}{49}
\end{aligned}
$$
\end{enumerate}
\pagebreak
\section{Question 4}
Let $X_T$ be the fraction of fully repaired machines given $T$ number of machines.\\
Where,
$$\begin{aligned}
X_T&=\frac{r_1+r_2+\cdots+r_T}{T}\\
\sum_{i=1}^T r_i&+\sum_{i=1}^T b_i=T\\
\end{aligned}$$
$$
r_i=\left\{\begin{array}{ll}
1&i\text{th machine is fully repaired}\\0& \text{otherwise}
\end{array}\right.
$$
Let $Y$ be the time to complete repairs on any given machine\\
Let $Z$ be the time since the start of repair for mistake to occur.
$$
\begin{aligned}
Y&\sim\text{Exp}(\mu)\\
Z&\sim\text{Exp}(\frac{1}{\lambda})\\
Z&\sim\text{Exp}(\lambda^{-1})\\
\Pr(Y<Z)=\frac{\mu}{\mu+\lambda^{-1}}\ \ \ \ &\#\ \text{difference of two exp. distributions}
\end{aligned}
$$
$$
r_i=\left\{\begin{array}{ll}
1&Y<Z\\0&Y>Z
\end{array}\right.
$$
$$
E[r_i]=\frac{\mu}{\mu+\lambda^{-1}}
$$
$$
b_i=\left\{\begin{array}{ll}
1&i\text{th machine is broken}\\0& \text{otherwise}
\end{array}\right.
$$
\\
The long-run fraction of fully repaired machines can be expressed by:
$$
\begin{aligned}
\lim_{T\rightarrow\infty}X_T&=\lim_{T\rightarrow\infty}\frac{\sum_i^T r_i}{T}\\
&=\frac{E[r_i]}{E[T_i]}\\
&=\frac{E[r_i]}{1}\ \ \ \ \#\ \text{machine will always be checked}\\
&=\frac{\mu}{\mu+\lambda^{-1}}
\end{aligned}
$$
\end{document}